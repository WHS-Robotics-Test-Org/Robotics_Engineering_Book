% Options for packages loaded elsewhere
\PassOptionsToPackage{unicode}{hyperref}
\PassOptionsToPackage{hyphens}{url}
%
\documentclass[
]{book}
\usepackage{lmodern}
\usepackage{amssymb,amsmath}
\usepackage{ifxetex,ifluatex}
\ifnum 0\ifxetex 1\fi\ifluatex 1\fi=0 % if pdftex
  \usepackage[T1]{fontenc}
  \usepackage[utf8]{inputenc}
  \usepackage{textcomp} % provide euro and other symbols
\else % if luatex or xetex
  \usepackage{unicode-math}
  \defaultfontfeatures{Scale=MatchLowercase}
  \defaultfontfeatures[\rmfamily]{Ligatures=TeX,Scale=1}
\fi
% Use upquote if available, for straight quotes in verbatim environments
\IfFileExists{upquote.sty}{\usepackage{upquote}}{}
\IfFileExists{microtype.sty}{% use microtype if available
  \usepackage[]{microtype}
  \UseMicrotypeSet[protrusion]{basicmath} % disable protrusion for tt fonts
}{}
\makeatletter
\@ifundefined{KOMAClassName}{% if non-KOMA class
  \IfFileExists{parskip.sty}{%
    \usepackage{parskip}
  }{% else
    \setlength{\parindent}{0pt}
    \setlength{\parskip}{6pt plus 2pt minus 1pt}}
}{% if KOMA class
  \KOMAoptions{parskip=half}}
\makeatother
\usepackage{xcolor}
\IfFileExists{xurl.sty}{\usepackage{xurl}}{} % add URL line breaks if available
\IfFileExists{bookmark.sty}{\usepackage{bookmark}}{\usepackage{hyperref}}
\hypersetup{
  pdftitle={Robotics Engineering},
  pdfauthor={Steve Cline},
  hidelinks,
  pdfcreator={LaTeX via pandoc}}
\urlstyle{same} % disable monospaced font for URLs
\usepackage{longtable,booktabs}
% Correct order of tables after \paragraph or \subparagraph
\usepackage{etoolbox}
\makeatletter
\patchcmd\longtable{\par}{\if@noskipsec\mbox{}\fi\par}{}{}
\makeatother
% Allow footnotes in longtable head/foot
\IfFileExists{footnotehyper.sty}{\usepackage{footnotehyper}}{\usepackage{footnote}}
\makesavenoteenv{longtable}
\usepackage{graphicx,grffile}
\makeatletter
\def\maxwidth{\ifdim\Gin@nat@width>\linewidth\linewidth\else\Gin@nat@width\fi}
\def\maxheight{\ifdim\Gin@nat@height>\textheight\textheight\else\Gin@nat@height\fi}
\makeatother
% Scale images if necessary, so that they will not overflow the page
% margins by default, and it is still possible to overwrite the defaults
% using explicit options in \includegraphics[width, height, ...]{}
\setkeys{Gin}{width=\maxwidth,height=\maxheight,keepaspectratio}
% Set default figure placement to htbp
\makeatletter
\def\fps@figure{htbp}
\makeatother
\setlength{\emergencystretch}{3em} % prevent overfull lines
\providecommand{\tightlist}{%
  \setlength{\itemsep}{0pt}\setlength{\parskip}{0pt}}
\setcounter{secnumdepth}{5}
\usepackage{booktabs}
\usepackage{amsthm}
\makeatletter
\def\thm@space@setup{%
  \thm@preskip=8pt plus 2pt minus 4pt
  \thm@postskip=\thm@preskip
}
\makeatother
\usepackage[]{natbib}
\bibliographystyle{apalike}

\title{Robotics Engineering}
\author{Steve Cline}
\date{2020-07-23}

\begin{document}
\maketitle

{
\setcounter{tocdepth}{1}
\tableofcontents
}
```
options(tinytex.verbose = TRUE)

\begin{center}\rule{0.5\linewidth}{0.5pt}\end{center}

\hypertarget{prerequisites}{%
\chapter{Prerequisites}\label{prerequisites}}

While we will use Canvas as much as possible in this course, it is not a suitable platform for a course which relies heavily on programming. As a result, you will need to become familiar with using sites like GitHub. Familiarity with these sites will prepare you for future work in college and beyond.

For this course you will need to do the following to prepare to learn:

Locate the \protect\hyperlink{course-lecture-notes}{Course Lecture Notes} section of this site. Browse through the notes to gain an understanding of what we will be covering this year.

Sign up for a GitHub account and fill out this form.

Bookmark the course GitHub Repository. This is where many of the programming resources will be found. In addition, assignments will be submitted here.

Locate our two textbooks for the course. They are both online and free. The first is an electronics textbook called Lessons in Electric Circuits. The second is a programming textbook called Think Python.

Subscribe to the course YouTube Channel.

Directions for how to do this are found in the \protect\hyperlink{follow-the-course-youtube-channel}{Follow the Course YouTube Channel} section of this site.

Log on to the class Jupyter Notebook server

Sign up for an OnShape account by following the instructions in the \protect\hyperlink{onshape-account-setup}{OnShape Account Setup} section of this site.

Complete the WHS Mechatronics Lab Code of Conduct Agreement.

\hypertarget{intro}{%
\chapter{Introduction}\label{intro}}

\hypertarget{a-brief-overview-of-the-course}{%
\section{A brief Overview of the Course}\label{a-brief-overview-of-the-course}}

Remember that engineers spend a majority of their time debugging their designs. You will too. It is not because this course is made too hard for you\ldots it is just the way engineering works.

``They say that no plan survives first contact with implementation. I'd have to agree.'' -- Mark Watney, Sol 40

Why begin with this quote? One of my favorite books and movies of all time is The Martian. The struggles of the main character, Mark Whatney, demonstrate the challenges facing engineers. While we will not be risking our lives on another planet, our solutions to the problems in this course, follow the same pattern as his. Where he sought to survive being the only human on another planet by engineering his way out, we will survive this course by learning to attack problems in the same way and remember not to give up. We must keep in mind, no plan will work the first time.

During this class students will explore the field of mechatronics using a variety of hands-on activities. Students begin the semester with an introduction to basic history and theory of robotics, the engineering process and tools and processes used to create robotic devices. We will introduce basic electronics concepts. Moving forward programming becomes an essential and vital element. Students program the onboard micro-processor found on a Raspberry Pi 3B+. This control board will use the Raspian OS which is a version of Linux. While students will work within the Linux shell, the programming language of this course is Python. Students work individually and in teams to design and build simple and complex mechatronic systems capable of meeting a variety of criteria including driving, pushing, controlling speed, etc. Sensors are introduced to allow robotic devices to interact with the environment. Actuator design is discussed and different manipulator systems are introduced. As an essential part of mechanical design, students will be exposed to CAD concepts using OnShape.

As you can see from this Euler diagram here, Mechatronics is a complicated field of engineering that combines many areas of study. This is the more technical name for robotics and was created by Tetsuro Mori in 1971 and has served as the name for this field since. In this course, you will be introduced to several of the areas on this diagram.

Since much of this course will be delivered remotely, be sure that you consider creating a good remote learning environment by doing the following:

Create a dedicated workspace for yourself that you can keep neat and organized throughout the year. Pick a place that is as quiet as possible so that you can concentrate. Do not use this space for anything else. In this course that space should include a place for your computer as well as an area to work on electronics projects - preferable with a hard surface like a desktop or workbench.

Manage your sleep schedule. You will need to be able to concentrate and stay focused since you do not have a teacher hovering over your workspace to keep you focused. If you go to bed at a consistent time and wake up at a consistent time, you will be able to learn more effectively.

Make sure you know how to log on to every account needed for the course. See the \protect\hyperlink{prerequisites}{Prerequisites} for all of the details for this course. Be sure to ask questions if you encounter any difficulties.

\hypertarget{history-of-robotics}{%
\section{History of Robotics}\label{history-of-robotics}}

Adapted from Robotics: A Brief History, Stanford University

Origins of ``Robot'' and ``Robotics''

The word ``robot'' conjures up a variety of images, from R2D2 and C3PO of Star Wars fame; to human-like machines that exist to serve their creators (perhaps in the form of the cooking and cleaning Rosie in the popular cartoon series the Jetsons); to the Rover Sojourner, which explored the Martian landscape as part of the Mars Pathfinder mission. Some people may alternatively perceive robots as dangerous technological ventures that will someday lead to the demise of the human race, either by outsmarting or outmuscling us and taking over the world, or by turning us into completely technology-dependent beings who passively sit by and program robots to do all of our work. In fact, the first use of the word ``robot'' occurred in a play about mechanical men that are built to work on factory assembly lines and that rebel against their human masters. These machines in R.U.R. (Rossum's Universal Robots), written by Czech playwright Karl Capek in 1921, got their name from the Czech word for slave.

The word ``robotics'' was also coined by a writer. Russian-born American science-fiction writer Isaac Asimov first used the word in 1942 in his short story ``Runabout.'' Asimov had a much brighter and more optimistic opinion of the robot's role in human society than did Capek. He generally characterized the robots in his short stories as helpful servants of man and viewed robots as ``a better, cleaner race.'' Asimov also proposed three ``Laws of Robotics'' that his robots, as well as sci-fi robotic characters of many other stories, followed:

A robot may not injure a human being or, through inaction, allow a human being to come to harm.

A robot must obey the orders given it by human beings except where such orders would conflict with the First Law.

A robot must protect its own existence as long as such protection does not conflict with the First or Second Law.

Early Conceptions of Robots

One of the first instances of a mechanical device built to regularly carry out a particular physical task occurred around 3000 B.C.: Egyptian water clocks used human figurines to strike the hour bells. In 400 B.C., Archytus of Taremtum, inventor of the pulley and the screw, also invented a wooden pigeon that could fly. Hydraulically-operated statues that could speak, gesture, and prophecy were commonly constructed in Hellenic Egypt during the second century B.C.

In the first century A.D., Petronius Arbiter made a doll that could move like a human being. Giovanni Torriani created a wooden robot that could fetch the Emperor's daily bread from the store in 1557. Robotic inventions reached a relative peak (before the 20th century) in the 1700s; countless ingenius, yet impractical, automata (i.e.~robots) were created during this time period. The 19th century was also filled with new robotic creations, such as a talking doll by Edison and a steam-powered robot by Canadians. Although these inventions throughout history may have planted the first seeds of inspiration for the modern robot, the scientific progress made in the 20th century in the field of robotics surpass previous advancements a thousandfold.

The First Programmable System

There is very little doubt today about both how essential programming is to robotics and who the earliest computer programmer was. While many would immediately think of modern computer scientists like Steve Wozniak or earlier programmers like Alan Turing, the first person to successfully put forth the idea of a programmable system like a computer was Ada Lovelace. In her article of 1842 (Yes, during the John Tyler administration) she proposed that numbers and objects could be used to ``express abstract scientific operations'' and create action based on rules. This was the first time anyone had proposed programming a device to act on inputs based on rules. This is the very heart of what it is to be a robot. Despite the great disadvantage of being born a woman in a very male dominated 19th century, she founded the field which would become the backbone of mechatronics.

The Origins of the Math of Robotics

The backbone of decision making in the mechatronics is a field of math called Boolean Algebra. As a professor of mathematics at Queen's College Cork in Ireland, George Boole published a work titled An Investigation of the Laws of Thought in 1854. Boole seeks to prove mathematically the existence of god through the application of logic. While his success at justifying the existence of god has not been supported, the underlying mathematical system that he invented would be used a century later as the foundation for modern day computers.\\
Without Boole's contribution to mathematics and logic, modern microprocessers built from transistors would not be possible. While he lived in obscurity during his life, the long legacy of George Boole is found in every computer programming language, including Python. The code below, which you will become familiar with this year, demonstrates this application.

\begin{verbatim}
  While True:
        print("Boolean Algebra makes computer decision making possible.")
\end{verbatim}

For a more in-depth understanding of his contributions to mechatronics see the embedded video in the \protect\hyperlink{supplemental-videos}{Supplemental Videos} Section.

The First Modern Robots

The earliest robots as we know them were created in the early 1950s by George C. Devol, an inventor from Louisville, Kentucky. He invented and patented a reprogrammable manipulator called ``Unimate,'' from ``Universal Automation.'' For the next decade, he attempted to sell his product in the industry, but did not succeed. In the late 1960s, businessman/engineer Joseph Engleberger acquired Devol's robot patent and was able to modify it into an industrial robot and form a company called Unimation to produce and market the robots. For his efforts and successes, Engleberger is known in the industry as ``the Father of Robotics.''
Academia also made much progress in the creation new robots. In 1958 at the Stanford Research Institute, Charles Rosen led a research team in developing a robot called ``Shakey.'' Shakey was far more advanced than the original Unimate, which was designed for specialized, industrial applications. Shakey could wheel around the room, observe the scene with his television ``eyes,'' move across unfamiliar surroundings, and to a certain degree, respond to his environment. He was given his name because of his wobbly and clattering movements. The \protect\hyperlink{supplemental-videos}{Supplemental Videos} section of this text has a video made about Shakey.

In order to better understand the field of mechatronics, you will choose one person who has been foundational to the development of the field.

Assignment: The Innovators Project

\hypertarget{electrical-concepts}{%
\chapter{Electrical Concepts}\label{electrical-concepts}}

This class and modern life relies heavily on electricity and electronics. Our course electronics text points this out quite eloquently:

The history of electricity starts more than two thousand years ago, with the Greek philosopher Thales being the earliest known researcher into electricity. But it was Alessandro Volta who created the most common DC power source, the battery (for this invention the unit Volt was named after him).

Direct current (also known as DC) is the flow of charged particles in one unchanging direction (most commonly found as electron flow through conductive materials). DC can be found in just about every home and electronic device, as it is more practical (compared to AC from power stations) for many consumer devices. Just a few of the places where you can find direct current are batteries, phones, computers, cars, TVs, calculators, and even lightning.

We will begin exploring a number of concepts relating to electricity as we will essentially be using electrons as our modes of information processing. From the electrons that flow through a button or distance sensor to be read by the microprocessor to those which supply the power for our motors and servos, we are heavily dependent on electricity. Before we get into this, we need to go over some very important concepts. Take out your Raspberry Pi and follow along with the video below.

Video here
Rules of this course in regards to electricity:

Always work only with DC electricity, never use AC voltage except for when you plug your Raspberry Pi into the outlet.

Always work on your circuits when the devices are unplugged and batteries removed. Not only is this a safety issue, it will prevent the need to replace electrical components

Only use the components listed in the laboratory activity on which you are working. Do not try new components you have never used before or are unfamiliar with.

Keep your work area clean and uncluttered. Do not keep beverages on your workstation.

Circuit lab setup

\hypertarget{circuits}{%
\section{Circuits}\label{circuits}}

A basic understanding of circuit electricity is essential going forward. For this section of the course, we will begin using our online textbook. It gives us an excellent overview of electrons, circuits and polarity. Please keep in mind that, in order to begin using the Raspberry Pi (i.e.~receiving the power supply) you will need to pass all of the required quizzes with at least a 70\%.

Key Concepts of Section 3.1

A circuit is a loop of conductive material.

In order for electrons to flow a closed circuit is required.

A circuit is open or broken if a complete path for the electrons to flow no longer exists.

Rather than me trying to do a better job than our textbook, you are going to read the short section on circuits. These questions will be similar to those that you will see on your unit exam (The one you need at least a 70\% on to move on). So\ldots that means that you are not completing this worksheet for a grade. Instead, you are trying to measure how well you learned the concepts in the reading. Much of this year's content will be delivered this way. Get your head around the idea of learning for the sake of learning.

To help you understand this concept, here is an excellent video from SparkFun Electronics on the topic.

Assignments for section 3.1:

After reading the section on circuits, complete the worksheet

Using (tool for this), design a closed circuit with the following components:

5V DC power supply

Red LED

330 Ohm resistor

Submit the (method of submission) here.

\hypertarget{voltage-and-current}{%
\section{Voltage and Current}\label{voltage-and-current}}

The movement of electrons from one pole to another is what we use in robotics to make our robots work. By altering the volume of electrons moving through the wires connected to a motor we can alter the speed of the wheel attached to the motor. The potential energy coming in to our microprocessor pin over a wire attached to a photoresistor tells us the level of brightness or lumosity in a room. Putting these two concepts together we could create a robotic system that opens the blinds on our lab when the sun sets in the west. That is, of course, if we can program our system to make that decision on its own. We will get to the programming part later. For now, we will examine the concepts of voltage and current which allow us, along with resistance, to receive information through sensor and send information out to an actuation device like a motor.

\hypertarget{resistance}{%
\section{Resistance}\label{resistance}}

\hypertarget{ohms-law}{%
\section{Ohm's Law}\label{ohms-law}}

\hypertarget{using-a-digital-multimeter}{%
\section{Using a Digital Multimeter}\label{using-a-digital-multimeter}}

This section will become active in the event that we are not using remote instruction.

\hypertarget{basic-soldering}{%
\section{Basic Soldering}\label{basic-soldering}}

This section will become active in the event that we are not using remote instruction.

\hypertarget{mechanical-design}{%
\chapter{Mechanical Design}\label{mechanical-design}}

Need to teach them a few things.

\hypertarget{basic-mechanics}{%
\section{Basic Mechanics?}\label{basic-mechanics}}

\hypertarget{advanced-mechanics}{%
\section{Advanced Mechanics?}\label{advanced-mechanics}}

\hypertarget{computer-aided-design}{%
\section{Computer Aided Design}\label{computer-aided-design}}

\hypertarget{python-programming}{%
\chapter{Python Programming}\label{python-programming}}

Finally! We are going to start learning how to make the Raspberry Pi control components like lights and motors and receive input from sensors. This is what will make your creations become real robots. For this unit we will refer to the course programming text quite often. Many of the programming problems come from the Downey text as well. For all of your assignments in this section you will start here though. Learn the basic concepts of that assignment, read more about it on the course text, go to GitHub Classroom to practice the program and finally, transfer what you have learned to the Raspberry Pi.

Let's get started by setting up the Pi and learning more about Python and it's wide range of applications.

\hypertarget{setting-up-the-raspberry-pi}{%
\section{Setting up the Raspberry Pi}\label{setting-up-the-raspberry-pi}}

These procedures are for students in the Robotics Engineering course at Windsor High School. As such, some of the initial steps required in setting up a Raspberry Pi have already been completed. If you are using a new, out of the box, Pi refer to one of the many tutorials online.

Equipment Needed:

Raspberry Pi 3B+

5V \textbar{} 2.5A power supply

32GB Micro SD card with pre-installed Linux based OS

Monitor

HDMI cable

Keyboard

Mouse

Optional: Windows, iOS or Linux Computer (not a Chromebook)

Connection to a LAN. In the lab this is ``Mechlab''. At home it is your router.

NOTE: For the initial setup we will use the monitor, keyboard and mouse. Eventually, we will only use the command line functions which will allow you to eliminate those components and control the Pi from your laptop remotely if you choose. If not, you can use the Pi as a computer itself but just enter the command line controls of the Linux terminal. Eventually, you will need to be able to do this though as we will disconnect the Pi from all physical I/O devices and make it the brains of a self-sufficient robotic system.

\hypertarget{putting-the-pi-together}{%
\subsection{Putting the Pi Together}\label{putting-the-pi-together}}

To begin, assemble the Pi by putting the SD card in the slot. Connect the monitor to the HDMI port, and the keyboard and mouse to the USB ports. Plug in the power cord and turn it on.

(Create a custom background for the RPi and install all of the programs needed for the class.)

\hypertarget{controlling-the-pi-with-ssh}{%
\subsection{Controlling the Pi with SSH}\label{controlling-the-pi-with-ssh}}

For many, it may be easiest just to use the Raspberry Pi as a standalone computer. In order to do this one just needs to keep the Pi connected to a monitor, keyboard and mouse. One of the main advantages of this is that one can just turn on the Pi, wait for the boot sequence to finish and start programming using the terminal. Another advantage is that other programming interfaces may be used such as Notepad++ which has built in code highlighting and markup functions that make programming in Python, and other languages, easier. More advanced users might even choose to install an interface like PyCharm which has even more functions aimed at enhancing the programming experience and may be connected to Git or an online code repository like GitHub. Maybe the biggest advantage of programming from the Pi desktop environment is that a web browser may be easily used alongside the command line terminal or programming interface.

Eventually though, we will all need to sever our connection to the peripherals. Imagine trying to test out a mobile robot system with the monitor still attached. When we make the switch we start interacting with the Pi through SSH (Secure Shell). This concept is not as difficult to understand as one might think. Essentially you are accessing the terminal window for the Pi through another computer using a piece of software like Putty. When both devices are connected to the same router, this is a fairly simple process. While we will not require this yet, it is strongly recommended that every student go over this process which is detailed in the section titled \protect\hyperlink{controlling-the-raspberry-pi-remotely}{Controlling the Raspberry Pi Remotely} in the last chapter of this book sometime in the first semester. The Capstone project will require this so we might as well get used to it now. Most will probably switch back and forth between using the Pi as a desktop computer and accessing it via SSH.

\hypertarget{the-parts-of-the-rpi-and-pinout}{%
\section{The parts of the RPi and Pinout}\label{the-parts-of-the-rpi-and-pinout}}

Let's examine the parts of this tiny computer that you will be using for the rest of the year\ldots it is really quite amazing how much can be packed on the small device.

There are several things worth noting here about the Pi:

While it is possible to power the device with a micro USB power supply that provides only 1A of current (older Android phones use this), it is best to use the full 2-2.5A power supply, especially when we start hooking up components to the GPIO pins

Be careful not to short two GPIO pins together or connect an external DC power supply incorrectly to one of the GPIO pins. These actions can damage essential components of the Pi.

TBC

A powerful feature of the Raspberry Pi is the row of GPIO (general-purpose input/output) pins along the top edge of the board. There are 40 pins which we can use for a wide range of robotics applications.
It is important to remember which pins are which as the locations do not represent the GPIO labels. If you forget or cannot find this diagram, use the command line in the terminal.

This is as good a place as any to start using the terminal. Before the introduction of the Apple Lisa with a graphical user interface (GUI), all computers used something like the terminal as the interface with humans. All computers still contain a terminal. The Raspberry Pi OS we are using is based on the Linux kernel so we will use Linux commands but the procedure is similar for other systems such as Windows and macOS. With the Raspberry Pi powered on press CTRL+ALT+T to open a terminal session.

After the terminal opens, type the following command: \texttt{pinout}

You should see something like this in the terminal:

Any of the GPIO pins can be designated (in software) as an input or output pin and used for a wide range of purposes. We will use them to send and receive electrical signals. Two 5V pins and two 3V3 pins are present on the board, as well as a number of ground pins (0V), which are unconfigurable. The remaining pins are all general purpose 3V3 pins, meaning outputs are set to 3V3 and inputs are 3V3-tolerant.

A GPIO pin designated as an output pin can be set to high (3V3) or low (0V). In addition pins GPIO12, GPIO13, GPIO18, GPIO19 may also be used with Pulse Width Modulation (PWM). A GPIO pin designated as an input pin can be read as high (3V3) or low (0V). This is made easier with the use of internal pull-up or pull-down resistors. Pins GPIO2 and GPIO3 have fixed pull-up resistors, but for other pins this can be configured in software.

It is possible to control GPIO pins using a number of programming languages and tools. For this course we will use Bash, the scripting language of Linux, and Python.

NOTE: Before you start your first programming assignment realize that in order to progress in this course you must fully learn the information in the text. If you skim or rush through you might be able to get many answers on the chapter quiz correct but you will miss many of the points. Programming is hard and learning to program is as well. It does not mean it is impossible. It just requires work. Take your time and make sure you are understanding what you read. If not, do the following:

Re-read the section.

Ask someone else in the class for clarification.

Google your question

Ask your instructor

This is good advice for all courses, especially the further you go in school. Many of you will go so far in college that you basically stop taking classes and just learn on your own for the sake of understanding something.

Assignment 5.2
For this assignment you will learn some basic commands and functions in Linux and how to find out how to use them.
Open up a terminal on your Raspberry Pi as before.

Type the following:
\texttt{help}

This is a list of all the built in functions that come with this version of Linux. As you add more packages and create your own functions this list will get larger. If you scroll to the top your see this message:
\texttt{Type\ \textquotesingle{}help\ name\textquotesingle{}\ to\ find\ out\ more\ about\ the\ function\ \textquotesingle{}name\textquotesingle{}.}
Let's try it. One of the functions you will use fairy often is \texttt{cd}.

Type \texttt{help\ cd} and answer the following questions from the information on the screen:

In your own words that you understand, what does this function do? (You may need to Google some of the terms like ``directory'' to understand what that term means in Linux)

If you type \texttt{cd\ Desktop} what do you think will happen?

Now call up the help information for the function \texttt{if}.
Why would we ever use this function? Again, you might need to do a little research to understand the vocabulary here.

Write down or type your answers so we can discuss them in class next time.

\hypertarget{why-python}{%
\section{Why Python?}\label{why-python}}

It is worth spending a little time discussing the reasons why Python will be the chosen operating system for this course. Prior to Fall 2020, the course utilized the Arduino UNO as the microprocessor for the robotics systems. The Arduino is also a versatile, fully programmable microcomputer with input and output pins capable of producing and detecting low DC voltages. It utilizes the C++ programming language. There were a few limitations, however, with the Arduino UNO.

The biggest advantage of the Pi is the programming interface. Since the Pi is essentially a small Linux computer, applications and environments that function on Linux also function on the Pi. The default scripting language on Linux is called Bash. When you called up the \texttt{pinout} for the Raspberry Pi earlier, you were using Bash. It is a powerful command line interface. Why not just use Bash instead of learning another language within the Linux shell? The main reason for this is Python's usefulness outside of this class. We know from past experience that almost all of our Robotics Engineering students will go on to some kind of technical field of study in college. While those do not all go in to robotics, they all need to learn new languages. Python is about the most useful language one can learn in high school if planning to go on to a technical degree in college like engineering, computer science, data science or machine learning as well as almost any natural science field of study. It is powerful, easy to learn and widely used. The most recent StackOverflow survey ranked Python 3rd most popular among users. By learning this language, you will be more prepared to take on the challenges of a field of study in college related to robotics. Since one of the main objectives of our program is to help students prepare for post-secondary STEM fields of study, Python is an easy choice.

Python is also an interpreted language instead of a compiled language. What is the difference? In order to get a compiled language to run on a computer, the additional step of compiling it into machine code must take place before the device will run the commands of the program. With an interpreted language, a program or virtual machine running on the device (In this case, the Python environment installed on the Raspberry Pi OS) interprets the script written by the programmer directly into machine code. This is a faster process than running a compiled language like C++. Other interpreted languages include Ruby, Java and Perl and good old BASIC (Which many of us adult programmers learned when we were younger on our Commodore 64 or Apple IIe computers). This structure makes Python faster to learn.

There are other advantages to the switch to the Pi. The first advantage that the Pi has over the UNO is an onboard wireless internet connection. The Pi can connect to other devices using either 2.4GHz or 5GHz frequencies as well as Bluetooth. The current version also includes a Power over Ethernet (PoE) capability. Making the switch to the Raspberry Pi will allow for more seamless IOT (Internet of Things) activities without the need for an additional board. Second, the Raspberry Pi can actually function like a computer. We can modify the OS easily by modifying the micro SD card used as RAM on the computer.

\hypertarget{hello-worldand-a-little-more}{%
\section{Hello, World\ldots and a Little More}\label{hello-worldand-a-little-more}}

It is a tradition in in programming to test out one's ability to program in any language by creating a program called ``Hello, World!''. This dates back to 1974 at Bell Labs. The actual C language code that was written was:

\texttt{main(\ )\ \{\ \ \ \ \ \ \ \ \ printf("hello,\ world\textbackslash{}n");\ \}}

In our case, we will do something very similiar. The first program we write will do the same as that one written in 1974. Open a terminal window by pressing CTRL+ALT+T. This will open a window that looks like this:

In order to run a command in Python through the terminal we need to set up the Python environment. Type this command into the terminal:

\texttt{python3}

This will show a message like this:

You know that you are able use Python commands when you get the \texttt{\textgreater{}\textgreater{}\textgreater{}} prompt. This is universal for all Python editors and the specific prompt for Python. We will try two methods of the ``Hello, World!'' program. First, we will use the print command. Type the following command into the interpreter:

\texttt{print\ ("Hello,\ World!")}

The quotation marks signify that we want to literally display what is directed by the \texttt{print} statement. Try it without the quotes to see what happens. This should produce something like this:

This is your first error message.

Assignment 5.4
Read the chapter from our text titled, ``Chapter 1 - The way of the program''. While you read the chapter open the Python environment through the Linux terminal on your Raspberry Pi as before. Follow along with the examples from the text as you go. This will help you learn Python more quickly. When you are done reading the chapter complete exercises 1 and 2 at the end of the chapter on your own to make sure you understand what you read. If not, see the Note above.

\hypertarget{functions}{%
\section{Functions}\label{functions}}

As our text states,

\begin{quote}
In the context of programming, a function is a named sequence of statements that performs a computation. When you define a function, you specify the name and the sequence of statements. Later, you can ``call'' the function by name.
\end{quote}

Let's take a look at this sequentially:

First, define a function. Second, call the function.

What does this look like?

Open the Python environment window and define your function like this:

Then we call the function like this:

So, what happened here? We defined the function \texttt{say\_hello()} . Then, we called the function and passed a parameter into the function. Can you tell what the parameter is?

Why do we need to know how to use functions?

The first reason is that the standard Python 3 library is filled with functions. When you type the command \texttt{print\ ("Hello,\ World!")} you are calling the built in Python \texttt{print()} function. As you read through the chapter on functions you will see that there are many other examples of useful functions packed into Python.

The second reason is to make your programs easier to use. For example, let's say you make a robot that needs to follow this pattern as it drives alone a course. There are many reasons why you might need to do something like this. Maybe the problem to solve is navigating around obstacles. Maybe the robot needs to map a facility that humans cannot get into because of hazardous conditions like radiation. Maybe your robotics teacher just wants you to demonstrate you can use functions. At any rate, functions will make your job as a programmer eaiser. Here is the path the robot will take:

How many times does the robot need to make a 90 degree left turn? (nine) Here is an example of how we could execute that turn using Python and our Raspberry Pi (Don't worry about the syntax now, we will get there.)

As you will learn, there are other commands that need to be used to set up this program but these four lines are what actually make the Raspberry Pi control the left and right wheels of the robot so that the right wheel moves forward for two seconds while the left wheel stays fixed. This will make the robot turn left. If the two seconds is the correct amount of time, it will make a 90 degree turn.

Think about writing the program to just make the robot drive around. You would need to replicate this code nine times. You probably would not retype it nine times. Instead you would copy and paste it. This is known as brute force programming. There are many reasons not to do this. The biggest reason has to do with the third line in our left turn program. What if two seconds is not exactly right? If that is a slight overturn then the robot will go off course with every turn which means we need to go back in and change that value on every \texttt{sleep()} function. What if the battery on the robot is at a different charge level from one day to the next? Again, we will need to change nine lines of code. Remember, we humans are not very good at repetitive tasks. That is what we use computers for - to do things over and over. Why not let the computer do the repetitive stuff?

Instead, let's make a function called \texttt{left\_turn()} like this:

And call it nine times throughout the program with the command \texttt{left\_turn(2)} . If we found that it actually takes three seconds instead of two to execute a 90 degree turn then we would just need to change the parameter we pass to the function, represented by the variable \texttt{sleep\_time}.

In that case, the call would look like this: \texttt{sleep\_time(3)}. We could even make things simpler yet by defining a variable at the beginning of the program to represent the number of seconds we want to sleep during the turn - something like \texttt{turn\_sleep\_time\ =\ 2} . Then, we would call the function like this:

\texttt{left\_turn(turn\_sleep\_time)}

Now, every time we change the variable \texttt{turn\_sleep\_time} we are changing the parameter passed to the function everywhere in the program. We have gone from changing nine different lines of code to changing just one. This gets really impressive when we think about all the right turns we have to make as well. See how functions are an integral part of robotics? Every modern programming language uses functions. Once you learn it for Python you can use it in other languages as well with only minor changes to syntax.

If we write programs that need to repeat a process nine times, functions save us a lot of work and potential errors. If we need our robot to do something nine hundred times or nine million times we have to use them. Now let's spend some time learning more about the functions built in to Python using our text:

Assignment 5.5
Read the chapter from our text titled, ``Chapter 3 - Functions''. While you read the chapter open the Python environment through the Linux terminal on your Raspberry Pi as before. Follow along with the examples from the text as you go. This will help you learn Python more quickly. When you are done reading the chapter complete exercises 1, 2 and 3 at the end of the chapter on your own to make sure you understand what you read. If not, see the Note above.

\hypertarget{conditionals}{%
\section{Conditionals}\label{conditionals}}

Besides completing repetitive tasks for us humans, robots need to be able to make decisions based on the condition of their environment. Later, we will learn how to use sensors to understand what is happening in a robot's environment. First, we will learn how conditional statements work in Python. If you have not already done so, it might be helpful to get a little background on computer logic. The inventor of the logical system we use in robotics was George Boole. There is a video in the \protect\hyperlink{supplemental-videos}{Supplemental Videos} section of this book about his contributions to the field.

For now, we will concern ourselves with some basic logical or Boolean operators. These are different than mathematical operators. When we write the following statement using mathematical operators:

\texttt{x\ =\ 6}

we are making a statement that the variable \texttt{x} is equal to the integer 6. By contrast, when we use the same operator (=) in logical math we are asking ``Is one equal to the other?'' We script that questions like this:

\texttt{x\ ==\ 6}

Notice that we use 2 = signs instead of one. This is not a typo. In fact, it is universal across many programming languages. Instead of declaring x is equal to 6, it is asking Is x equal to 6? What are the possible answers to this question? For the answer, read section 5.2 of our text.

Notice that there are many other logical (aka relational) operators. We can use these operators which return a value of true or false to help our robots make decisions. To do this we use functions like the \texttt{if()} statement. Let's try it out using another Python function \texttt{len()}. We will create a function that determines if a word is longer than seven letters. If so, a message will be displayed saying so. If not, a different message will be displayed. Before you read how to do this, see if you can figure it out. Follow these steps using the textbook as a guide:

Define your function and pass a parameter through that is the word you are going to test.

Write a conditional statement using the \texttt{len()} Python function to determine the length of the word.

Use the \texttt{print()} function to return a statement that says if it is or is not a long word (greater than seven letters long).

See if you can include the word in the message returned by the function

Here is one way to do it:

Notice that we pass the word to the function by using the parameter \texttt{word} which then stores the string representing our word as a variable. The if statement checks the length of the word using the \texttt{len()} function which returns an integer equal to the number of alphanumeric characters in a string (our word). If that value is greater than or equal to nine, one message is printed. Since the only other possibility for the length of this word (if it is not greater than or equal to nine) is that it is less than nine letters, we can use the \texttt{else} function to print a different message. Try it out now by typing this function into the terminal under the Python environment and calling the function using different words. You will probably get some ``Traceback'' errors the first time you do but stick with it. When you are done compare your result to the results below:

Think of the possibilities now. What could your robot do with this new power! If we go back to our example in the previous section in which our robot had to navigate around obstacles by turning 90 degrees at a time we can see some interesting solutions. If we could use a compass sensor to determine the angle of our robot in relation to magnetic north, we could use conditional statments to turn the robot. Instead of trying to figure out exactly how long it takes for the robot to turn 90 degrees, we could write a script that waited until the angle in relation to magnetic north changed by 90 degrees. We could even do this with one function. It might look something like this:

This is a lot to take in for a new programmer so let's briefly go over this function. When it is called we put in a value for direction. The possible correct values are ``left'' and ``right''. If the direction is ``right'' the left wheel moves until the angle of the robot is equal to the desired angle. If ``right'' is passed to the function the opposite happens. Just in case the human mispells the word or puts something else in a third print statement is used to tell them that their input is not valid. We will get to this process later in the course.

Assignment 5.6
Read the chapter from our text titled, ``Chapter 5 - Conditionals''. While you read the chapter open the Python environment through the Linux terminal on your Raspberry Pi as before. Follow along with the examples from the text as you go. This will help you learn Python more quickly. When you are done reading the chapter complete exercises 1, 2 and 3 at the end of the chapter on your own to make sure you understand what you read. If not, see the Note above.

\hypertarget{iteration}{%
\section{Iteration}\label{iteration}}

\hypertarget{robotic-types}{%
\chapter{Robotic Types}\label{robotic-types}}

Discuss types of robotic systems.

\hypertarget{databot}{%
\section{DataBot}\label{databot}}

\hypertarget{mobilebot}{%
\section{MobileBot}\label{mobilebot}}

\hypertarget{solarbot}{%
\section{SolarBot}\label{solarbot}}

\hypertarget{capstone-project}{%
\chapter{Capstone Project}\label{capstone-project}}

\hypertarget{going-further-with-robotics}{%
\chapter{Going Further with Robotics}\label{going-further-with-robotics}}

\hypertarget{robotics-in-higher-education}{%
\section{Robotics in Higher Education}\label{robotics-in-higher-education}}

\hypertarget{supplemental-videos}{%
\section{Supplemental Videos}\label{supplemental-videos}}

The Genius of George Boole

Shakey:Experiments in Robotic Planning and Learning (1972)

\hypertarget{course-materials-and-instructions}{%
\chapter{Course Materials and Instructions}\label{course-materials-and-instructions}}

\hypertarget{course-lecture-notes}{%
\section{Course Lecture Notes}\label{course-lecture-notes}}

\hypertarget{follow-the-course-youtube-channel}{%
\section{Follow the Course YouTube Channel}\label{follow-the-course-youtube-channel}}

\hypertarget{using-github}{%
\section{Using GitHub}\label{using-github}}

\hypertarget{think-python-textbook}{%
\section{Think Python Textbook}\label{think-python-textbook}}

\hypertarget{all-about-circuits-textbook}{%
\section{All About Circuits Textbook}\label{all-about-circuits-textbook}}

\hypertarget{download-and-install-pycharm}{%
\section{Download and Install PyCharm}\label{download-and-install-pycharm}}

\hypertarget{jupyter-notebooks}{%
\section{Jupyter Notebooks}\label{jupyter-notebooks}}

\hypertarget{onshape-account-setup}{%
\section{OnShape Account Setup}\label{onshape-account-setup}}

\hypertarget{controlling-the-raspberry-pi-remotely}{%
\section{Controlling the Raspberry Pi Remotely}\label{controlling-the-raspberry-pi-remotely}}

Using a remote shell, one can control a Raspberry Pi with another computer. You can do this through the linux command line or with a GUI using a service like Tight VNC. Learning how to do this will allow our robots to become less tethered to a specific location. This concept, more broadly is known as the Internet of Things (IoT). Devices with internet connections can interact with one another over that connection. Many of you probably already have some IoT devices in your homes like thermostats that interact with each other to maintain the most energy-efficient comfort in your house to security cameras that may be accessed on a mobile device from anywhere. In short, our robots will not be fully robots until we learn to do this. As a result, the Capstone project for this course in semester two requires this.

In all three of the cases listed below one thing is assumed. The Raspberry Pi and the device using the Secure Shell are connected to the same router. In our lab, we will use the MechLab wifi network. At home, students will need to make sure to connect the pi to their home router which the computer is using.

This process will work for a computer running Windows, MacOS or Linux. It will not work for a Chromebook.

\hypertarget{setting-up-ssh-for-windows}{%
\subsection{Setting up SSH for Windows}\label{setting-up-ssh-for-windows}}

Complete the following steps to create a secure shell connection with a computer running Windows:

Be sure the Raspberry Pi is turned on and connected to router.

Be sure the computer is connected to the same network.

Open a terminal window on the Pi and type \texttt{ip\ address\ show}.

You will see something like this:

Make a note of this address, this is the address of the Pi on the network. The computer will also have an ip address on the network. To see that address open Windows PowerShell and type \texttt{ipconfig}. The ``IPv4 Address'' is the address of the computer. If both are on the same network, the first three numbers should be the same. If not, they are not connected to the same network and a SSH connection cannot be made.

Download and install Putty (Most will choose the 64 bit Windows Installer file).

Open Putty.

In the ``Host Name (IP address)'' box, type in the address of the Raspberry Pi.

Click ``Yes'' at the warning.

Enter the username and password for the device.

You are now connected to the Pi via SSH to the command line.

\hypertarget{setting-up-ssh-for-mac}{%
\subsection{Setting up SSH for Mac}\label{setting-up-ssh-for-mac}}

\hypertarget{setting-up-ssh-for-linux}{%
\subsection{Setting up SSH for Linux}\label{setting-up-ssh-for-linux}}

  \bibliography{book.bib,packages.bib}

\end{document}
